%% usamos: Termo1-Termo2-{...}-TermonN      para siglas fechadas, ou seja, valores ou 
%%                                          passos de um método onde todos são necessários
%%         Termo1, Termo2, {...}, {TermoN}  para siglas aberts, ou seja, valores ou
%%                                          passos de um método onde alguns são opcionais
%%         Termo1 Termo2 {...} TermoN       para nomes de pessoas ou lugares
%%
%% a sigla é repetida em negrito, seguido da tradução em português (em um dos formatos acima), 
%% seguida da tradução em inglês italizada

\newacronym{IDC}{IDC}{Coporação de Dados Internacional - do inglês: ``\textit{International Data Corporation}'' - é um provedor de inteligência de mercado, serviços de consultoria, e eventos de tecnologia da informação, telecomunicações, e mercados de consumo tecnológicos, sediada em Needham, MA, Estados Unidos}

\newacronym{IDS}{IDS}{Sistema de Detecção de Intrusos - do inglês: ``\textit{Intrusion Detection System}'' - é um dispositivo ou \textit{software} monitora uma rede ou sistemas buscando e alertando os operadores na ocorrência de atividades maliciosas ou violações de políticas de segurança}

\newacronym{IP}{IP}{Endereço Protocolo Internet - do inglês: ``\textit{Internet Protocol Address}'' - é um código numérico assinalado a cada dispositivo conectado à rede Internet}

\newacronym{ICMP}{ICMP}{Protocolo de Controle de Mensagens Internet - do inglês: ``\textit{Internet Control Message Protocol}'' - é um protocolo utilizado por dispositivos de rede para envio de mensagens de rede e informações operacionais}

\newacronym{GRE}{GRE}{Encapsulamento de Roteamento Genérico - do inglês: ``\textit{Generic Routing Encapsulation}'' - é o protocolo de tunelamento desenvolvido pela Cisco Systems (fabricante de dispositivos de rede) que permite embutir vários outros protocolos dentro de conexões de rede ponto-a-ponto.}

\newacronym{NAT}{NAT}{Tradução de endereço de Rede - do inglês: ``\textit{Network Address Translation}'' - é um método de mapeamento de endereço de rede em outro modificando informações do cabeçalho {\glsentryshort{IP}}}

\newacronym{PCA}{PCA}{Análise do Componente Principal - do inglês: ``\textit{Principal Component Analysis}'' - é o processo de computar os componentes principais e utilizá-los para executar uma mudança de base (mudança relativa de coordenadas no espaço) dos dados.}

\newacronym{TCP}{TCP}{Protocolo de Controle de Transmissão - do inglês ``\textit{Transmission Control Protocol}'' - um dos principais protocolos do conjunto implementado pela Internet que provê transmissão de cadeias de dados de forma ordenada, com correção de erros e assegurando a entrega}

\newacronym{UDP}{UDP}{Protocolo de Datagrams do Usuário - do inglês ``\textit{User Datagram Protocol}'' - um dos principais protocolos do conjunto implementado pela Internet e utilizado para transmitir mensagens de aplicações}

\newglossaryentry{bit}{name={bit},description={Unidade básica de informação em computação e comunicação digital, da abreviação do inglês - ``\textit{binary digit}'' - e que representa um estado lógico de dois possíveis valores, comumente sendo ``0'' ou ``1''}}

\newglossaryentry{byte}{name={byte},description={Unidade de informação digital comumente composta por oito {\glsentryplural{bit}}}}

\newacronym{MTU}{MTU}{Unidade de Transmissão Máxima - do inglês: ``\textit{Maximum Transmission Unit}'' - representa a maior unidade de dados do protocolo de comunicação que pode ser comunicada numa simples transmissão}

\newacronym{IETF}{IETF}{Força Tarefa de Engenharia da Internet - do inglês: ``\textit{Internet Engineering Task Force}'' - uma organização de padrões abertos que trabalha voluntariamente para padronizar a adoção do protocolo TCP/IP na Internet}

\newacronym{IEEE}{IEEE}{Instituto dos Engenheiros de Elétrica e Eletrônica - do inglês: ``\textit{Institute of Electrical and Electronics Engineers}'' - associação profissional de engenheiros de eletrônica e elétrica e disciplinas associadas}